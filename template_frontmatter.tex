%
%
% UCSD Doctoral Dissertation Template
% -----------------------------------
% http://ucsd-thesis.googlecode.com
%
%


%% REQUIRED FIELDS -- Replace with the values appropriate to you

% No symbols, formulas, superscripts, or Greek letters are allowed
% in your title.
\title{Accelerating Data Movement at Different Granularities in Datacenters}

\author{Shu-Ting Wang}
\degreeyear{\the\year}

% Master's Degree theses will NOT be formatted properly with this file.
\degreetitle{Doctor of Philosophy}

\field{Computer Science}
%\specialization{Anthropogeny}  % If you have a specialization, add it here

\chair{Professor Steven Swanson}
% Uncomment the next line iff you have a Co-Chair
% \cochair{Professor Cochair Semimaster}
%
% Or, uncomment the next line iff you have two equal Co-Chairs.
%\cochairs{Professor Chair Masterish}{Professor Chair Masterish}

%  The rest of the committee members  must be alphabetized by last name.
\othermembers{
Professor George C. Papen\\
Professor Geoffrey M. Voelker\\
Professor Jishen Zhao\\
}
\numberofmembers{4} % |chair| + |cochair| + |othermembers|


%% START THE FRONTMATTER
%
\begin{frontmatter}

%% TITLE PAGES
%
%  This command generates the title, copyright, and signature pages.
%
\makefrontmatter

%% DEDICATION
%
%  You have three choices here:
%    1. Use the ``dedication'' environment.
%       Put in the text you want, and everything will be formated for
%       you. You'll get a perfectly respectable dedication page.
%
%
%    2. Use the ``mydedication'' environment.  If you don't like the
%       formatting of option 1, use this environment and format things
%       however you wish.
%
%    3. If you don't want a dedication, it's not required.
%
%
\begin{dedication}
  To Olivia, my mom, and my late father
\end{dedication}


% \begin{mydedication} % You are responsible for formatting here.
%   \vspace{1in}
%   \begin{flushleft}
% 	To me.
%   \end{flushleft}
%
%   \vspace{2in}
%   \begin{center}
% 	And you.
%   \end{center}
%
%   \vspace{2in}
%   \begin{flushright}
% 	Which equals us.
%   \end{flushright}
% \end{mydedication}



%% EPIGRAPH
%
%  The same choices that applied to the dedication apply here.
%
% \begin{epigraph} % The style file will position the text for you.
%   \emph{A careful quotation\\
%   conveys brilliance.}\\
%   ---Smarty Pants
% \end{epigraph}

% \begin{myepigraph} % You position the text yourself.
%   \vfil
%   \begin{center}
%     {\bf Think! It ain't illegal yet.}
%
% 	\emph{---George Clinton}
%   \end{center}
% \end{myepigraph}


%% SETUP THE TABLE OF CONTENTS
%
\tableofcontents
\listoffigures  % Comment if you don't have any figures
\listoftables   % Comment if you don't have any tables



%% ACKNOWLEDGEMENTS
%
%  While technically optional, you probably have someone to thank.
%  Also, a paragraph acknowledging all coauthors and publishers (if
%  you have any) is required in the acknowledgements page and as the
%  last paragraph of text at the end of each respective chapter. See
%  the OGS Formatting Manual for more information.
%
\begin{acknowledgements}
%
I want to thank Olivia, my significant other and soon-to-be lifelong partner, for supporting me through my PhD journey.
%
It is a wild ride and thank you to be always on my side.

I want to thank all faculty members who invested their time mentoring me, 
%
Prof. George Porter for the first four years of advising that significantly shapes my taste and capability to do research, 
%
Prof. Steve Swanson for chairing my committee and giving me the last push to finish the dissertation,
%
Prof. Geoff Voelker for being on my committee and his wonderful operating system class,
%
Prof. George Papen for being on my committee and his insights on networking and optics, 
%
Prof. Alex Snoeren on all the critiques on my figures during weekly meetings, 
%
and Prof. Jishen Zhao for being willing to serve on my committee.

To Rajdeep, Yibo, Nishant, Stew, Audrey, Anil, Ariana, Alex, and all other tenants of room 3140, thanks for hanging out with me and being my friends.
%
The camaraderie is invaluable, and I will bear that in mind when I embark on my next journey.

To Rohan, Byung Hoon, and Amin, thanks for all the long talks and chats about research and the time of questioning of our life choices while still working on research. 

To Pierre-Louis, Weitao, and Dan, thanks for sharing my research interest on CXL and other topics.
%
I really enjoy all the online chats and meetings and the idea of I am not in this alone pushes me to finish the last chapter of this dissertation.

To Dr. Yiting Xia, Jialong, Yiming, and Federico, thanks for hosting me at MPI-INF in Saarbr\"{u}cken while I am working on this dissertation. 

To \textit{Fujimak}, \textit{SmallGGRen}, \textit{HakkaFish} and \textit{StanfordSneaky}, I really enjoy our time on Mario cart racing and chatting about lives and politics. I will keep your real names in private in case other people scoop you folks from me. 

I want to thank Yin Chin Foundation for their scholarship helping me for conference travel and supporting myself during finical instable times.

Coffee as the fuel for any research output is essential. I want to thank the great coffee shops and roasters in San Diego: Bird Rock coffee, the Art of Espresso, and Finjin. 

At last, I want to thank all my collaborators and co-authors, who are listed next.
%
Chapter~\ref{daronpon:chap}, in part, reprints material as it appears in a draft titled: 
"Daronpon: Datacenter-scale Sub-RTT Replica Selection for Low-latency Applications"
by Shu-Ting Wang, Stewart Grant, Keerthana Ganesan, George Porter, and Alex C. Snoeren.
%
The dissertation author was the primary researcher and author of this material.

Chapter~\ref{fianchetto:chap}, in part, reprints material as it appears in a paper titled: "Data Motion Acceleration: Chaining Cross-Domain Multi Accelerators"
by Shu-Ting Wang, Hanyang Xu, Amin Mamandipoor, Rohan Mahapatra, Byung Hoon Ahn, 
Soroush Ghodrati, Krishnan Kailas, Mohammad Alian, and Hadi Esmaeilzadeh~\cite{dmx:hpca:2024}.
% 
The dissertation author was the primary researcher and author of this material.

Chapter~\ref{aurelia:chap}, in part, reprints material as it appears in a published WORD'23 workshoppaper titled: 
"Aurelia: CXL Fabric with Tentacle" by Shu-Ting Wang and Weitao Wang~\cite{aurelia:words:2023}. 
%
The dissertation author was the primary researcher and author of this material.
%
\end{acknowledgements}


%% VITA
%
%  A brief vita is required in a doctoral thesis. See the OGS
%  Formatting Manual for more information.
%
\begin{vitapage}
\begin{vita}
  \item[2013] B.~S. in Computer Science, National Tsing Hua University, Taiwan
  \item[2015] M.~S. in Computer Science, National Tsing Hua University, Taiwan
  \item[2016] Information System Technician, Civil Service Training and Protection Commission, Taiwan 
  \item[2017] Research Assistant, National Taiwan University, Taiwan
  \item[2021] Hardware Systems Foundation Engineer Intern, Meta
  \item[2024] Visiting Ph.~D. student, Max Planck Institute for Informatics, Germany
  \item[2017-2024] Ph.~D. in Computer Science, University of California San Diego
\end{vita}
\begin{publications}
  \item Rohan Mahapatra, Soroush Ghodrati, Byung Hoon Ahn, Sean Kinzer, \textbf{Shu-Ting Wang}, Hanyang Xu,  Lavanya Karthikeyan, Hardik Sharma,  Amir Yazdanbakhsh,  Mohammad Alian, Hadi Esmaeilzadeh, “In-Storage Domain-Specific Acceleration for Serverless Computing” in Proceedings of the 29th ACM International Conference on Architectural Support for Programming Languages and Operating Systems, Volume 2 (ASPLOS), 2024
  %
  \item \textbf{Shu-Ting Wang}, Hanyang Xu, Amin Mamandipoor, Rohan Mahapatra, Byung Hoon Ahn, Soroush Ghodrati, Krishnan Kailas, Mohammad Alian, and Hadi Esmaeilzadeh, “Data Motion Acceleration: Chaining Cross-Domain Multi Accelerators” in Proceedings of 2024 IEEE International Symposium on High-Performance Computer Architecture (HPCA), 2024
  %
  \item \textbf{Shu-Ting Wang} and Weitao Wang, “Aurelia: CXL Fabric with Tentacle,” in Proceedings of the 4th Workshop on Resource Disaggregation and Serverless (WORDS), 2023
  %
  \item Rohan Mahapatra, Byung Hoon Ahn, \textbf{Shu-Ting Wang}, Hanyang Xu, and Hadi Esmaeilzadeh, “Exploring Efficient ML-based Scheduler for Microservices in Heterogeneous Clusters,” in Proceedings of 2022 MLArchSys Workshop, 2022
\end{publications}
\end{vitapage}


%% ABSTRACT
%
%  Doctoral dissertation abstracts should not exceed 350 words.
%   The abstract may continue to a second page if necessary.
%
\begin{abstract}
  The dissertation investigates redundant communication between servers for large-scale web and cache requests and redundant data movement between accelerators for compute-intensive applications. 
  %
  Redundancy is an impending and critical issue for data centers designed for hardware accelerators and disaggregated resources. 
  %
  The dissertation makes the following three contributions to address this. The first contribution of the dissertation is \daronpon. 
  %
  \daronpon dynamically load-balances and reroutes large-scale requests of web and cache applications on a microsecond timescale. 
  %
  \daronpon prevents these requests, stranded on busy servers with network congestion and long queuing delays, from being processed. Daronpon shows improvement in various service time characterizations of different applications.
  %
  The second contribution of the dissertation is \dmx. 
  %
  \dmx acts as a compute-enabled bypass for inter-accelerator communication. 
  %
  \dmx accelerates the data restructuring needed between accelerators and saves the data movement between accelerators and CPUs for compute-intensive applications.
  %
  \dmx shows improvement in a series of benchmarks involving different application domains.
  %
  The third contribution of the dissertation is \aurelia. 
  %
  \aurelia leverages the emerging interconnect of CXL to investigate the design of a scalable fabric for accelerators and fabric-attached memory expansion.
  %
  \aurelia improves routing and transport based on the current specification of CXL and shows performance improvement on machine learning and key-value store applications.
\end{abstract}


\end{frontmatter}
