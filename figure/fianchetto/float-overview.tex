\begin{figure}[t]
\centering
\begin{subfigure}[b]{\columnwidth}
\includegraphics[width=\textwidth]{ASPLOS23/figures-resubmit/cpu-overview.pdf}
\caption{Current multi-accelerator systems.}
\label{fig:overview:current}
\end{subfigure}
%
\hspace{0.5in}
\begin{subfigure}[b]{\columnwidth}
\includegraphics[width=\textwidth]{ASPLOS23/figures-resubmit/dmx-overview.pdf}
\caption{Multi-accelerator systems with \dmx.}
\label{fig:overview:dmx}
\end{subfigure}
%
\caption{Current multi-acceleration systems rely on CPU for accelerator chaining. (a) shows a system with four heterogeneous accelerator cards. The CPU needs to intervene in the communication between accelerator cards. This involves data copies from system memory to accelerator memory and non-trivial data transformations. (b) The proposed \dmx framework removes the CPU from the data path of multi-acceleration. \dmx delivers the performance of a monolithic accelerator while offering the composability and programmablity of the baseline system.}
\vspace{-3ex}

\label{fig:overview}
\end{figure}
